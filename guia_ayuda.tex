\documentclass[]{article}


%opening
\title{Guía de creacion de Kernel - TP ARQUI - ITBA}
\author{Lpinilla}

\begin{document}

\maketitle

\begin{abstract}
	Este documento es una guía para ayudar en el desarrollo del\
	kernel en el Trabajo Práctico de la materia Arquitectura de\
	las computadoras. 
\end{abstract}

\subsection*{Consideraciones previas}
Las versiones de Ubuntu mayores a 16.04 presentan problemas al querer manipular la IDT, se recomienda utilizar Docker \textbf{\underline{desde cero.}}

\section*{PIC (Programmable Interface Controller)}
Lo primero que hay que entender es que las computadoras tienen 2\
PIC. El "MasterPIC" y el "SlavePIC". el slave esta "colgado" del\
master (cascading), por lo tanto se puede deshabilitar o ignorar\
lo que venga de él desde el master (IRQ02).\\
Cada PIC tiene 2 entradas, una para leer los datos y otra para conf
igurarlos, el master esta en 0x21 y 0x20. El slave en 0xA1h y 0xA0
respectivamente.

\section*{IRQ}
Cada PIC tiene 8 entradas, 1 bit por cada entrada $\longrightarrow$ 2bytes para configurarlo.

Las entradas del IRQ son:
\begin{center}
	\begin{tabular}{ |c|c| }
		\hline
		IRQ & Descripción\\
		\hline
		\multicolumn{2}{|c|}{MasterPIC} \\	
		\hline	
		0 & Timer-Tick\\
		1 & Keyboard \\
		2 & Cascade (used internally by the two PICs. never raised) \\
		3 & COM2 (if enabled)  \\
		4 & COM1 (if enabled)  \\
		5 & LPT2 (if enabled)  \\
		6 & Floppy Disk  \\
		7 & LPT1 / Unreliable "spurious" interrupt (usually)  \\
		\hline
		\multicolumn{2}{|c|}{SlavePIC} \\	
		\hline	
		8 & RTC \\
		9 & Free for peripherals / legacy SCSI / NIC  \\
		10 & Free for peripherals / SCSI / NIC  \\
		11 & Free for peripherals / SCSI / NIC  \\
		12 & PS2-Mouse \\
		13 & FPU / Coprocessor / Inter-processor  \\
		14 & Primary ATA Hard Disk  \\
		15 & Secondary ATA Hard Disk  \\
		\hline
	\end{tabular}
\end{center}

\section*{Masking del PIC}
Para cambiar la configuración del PIC tenemos que hacer un masking de bits.
Para deshabilitar algo, tenemos que ponerlo en 1, en caso contrario esta habilitado.

Supongamos que queremos solamente habilitar el teclado, para eso habría que
solamente dejar en 0 el bit 1 del MasterPic. Para eso debemos encontrar el valor
en hexa que deja al primer byte del pic en 1101.

Recordando que cada PIC tiene 2 bytes para configurarlo, tenemos que hallar el valor.

En este caso, queremos que este todo en 1 menos el anteúltimo bit: 1111-1101.
La primer parte corresponde al valor F mientras que la segunda al D. Por lo tanto,
el valor a mandarle al MasterPIC es FD.

\begin{verbatim}
	picMasterMask:
	push rbp
	mov rbp, rsp
	mov ax, di
	out	21h,al
	pop rbp
	retn
\end{verbatim}

Análogamente configuramos el slavePIC con el valor FF para no habilitar ninguna
otra interrupcion.

\section*{Interrupciones de Hardware}

Recordando que una interrupci\'on es un evento externo que ocurre, veamos como podemos
realizar driver de un periférico específico (el teclado).

\subsection*{Cadena de ejecuci\'on de interrupciones}
Veamos como resuelve la pc cuando se interact\'ua con el teclado.

\begin{center}
	\begin{enumerate}
		\item Se aprieta una tecla en el teclado.
		\item El teclado activa una interrupci\'on al PIC.
		\item El PIC recibe la interrupci\'on desde el IRQ01 y se
		fija si la deja pasar o no.
		\item Si la deja pasar, le indica al procesador que tiene una interrupci\'on.
		\item El procesador le indica si esta listo o no para recibir interrupciones.
		\item Si esta listo, le env\'ia, el PIC le env\'ia cual de sus interrupciones se activ\'o.
		\item El microprocesador con esa informaci\'on va a buscar a la IDT el registro correspondiente a la interrupci\'on IRQ01.
	\end{enumerate}
\end{center}

En la IDT, el PIC est\'a mapeado directamente, osea que IRQ0 arranca en X0h, IRQ1 en X1h, ... donde X en principio es 0\
Pero acá hay un problema ya que las primeras 32 entradas de la IDT son excepciones
por lo tanto se pisar\'ian las entradas, por eso, se "mueve" el inicio de las IRQ.
En este caso ser\'ia simplemente que X valga 2.\\

Por lo tanto, la tabla de los IRQ arranca en 20 (32 en hexa).

\subsection*{Creaci\'on de Interrupciones}
Para crear un driver tenemos que manejar las interrupciones del perif\'erico (ej Teclado).
Para eso, tenemos que hacer un par de cosas:

\begin{center}
	\begin{enumerate}
		\item Crear la entrada en la IDT.
		\item En la entrada de la IDT, asignar un puntero a funci\'on que va a ser la rutina
		de ejecuci\'on de la interrupci\'on.
		\item De ser necesario, llamar desde la rutina a una funci\'on en C.
	\end{enumerate}
\end{center}

Por ejemplo: Se crea la interrupci\'on en la entrada 21h. $\longrightarrow$ la interrupci\'on llama a la rutina de asm $\longrightarrow$ la rutina llama a una funci\'on de C que se encarga de interactuar con la lectura del teclado.\\

Hay que entender que la rutina apuntada por la IDT no es lo que finalmente ser\'a lo que lee del teclado, esta rutina llama a una funci\'on de C que se va a encargar de eso y de proveer m\'as funcionalidades.

\section*{Entradas comunes}
Entradas m\'as utilizadas cuando se lee en asm.
\begin{center}
	\begin{tabular}{ |c|c| }
		\hline
		Entradas & Dispositivo\\
		\hline	
		20-21h & MasterPIC\\
		A0-A1h & SlavePIC\\
		60-64h & Keyboard\\
		\hline
	\end{tabular}
\end{center}

	

\end{document}
